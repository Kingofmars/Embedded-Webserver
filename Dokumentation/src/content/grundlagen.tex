\chapter{Grundlagen}

Bevor wir mit dem eigentlichen Projekt beginnen konnten, 
mussten wir zuerst einmal grundlagenvorschung betreiben.
Als Problem tat sich heraus, das einige Team Mitglieder
noch sehr unerfahren in der Welt der Microcontroller war.

Nachdem wir in der ersten Besprechung die eigentliche Platine bekommen hatten,
konnte die erste Einarbeitung in den Controller beginnen.
Die Anfangsschwierigkeit lag darin, mit den angaben klar zu kommen 
und welche ein und Ausgänge wie benutzt werden konnten, mit der 
standardmäßig beigelegten Software.

Als erstes Problem trat auf, das wir nicht die richtigen für die
Entwicklung benötigten Geräte besaßen.
So hatten wir zwar einen \textbf{In-System-Programmer} (ISP) zum 
Anschluss der Platine an den PC doch war bei diesem Entwicklungswerkzeug
die falsche Pinbelegung vorhanden.

Doch hier konnten einige Anleitungen im internet Abhilfe schaffen

Des weiteren muss neben dem Controller noch weiteres Zubehör angeschafft werden,
damit das Projekt ausreichend getestet werden kann.


